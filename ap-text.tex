\ProvidesFile{ap-text.tex}[2021-01-04 text appendix]

\begin{VerbatimOut}{z.out}
\chapter{TEXT}


\section{Frenchspacing}

The ``\verb+\frenchspacing+'' command puts one space after sentences.
Normally two spaces are put after sentences.

{\frenchspacing
Lorem ipsum dolor sit amet, consectetuer adipiscing elit. Nam
cursus. Morbi ut mi. Nullam enim leo, egestas id, condimentum at,
laoreet mattis, massa. Sed eleifend nonummy diam. Praesent mauris
ante, elementum et, bibendum at, posuere sit amet, nibh. Duis
tincidunt lectus quis dui viverra vestibulum. Suspendisse
vulputate aliquam dui. Nulla elementum dui ut augue. Aliquam
vehicula mi at mauris. Maecenas placerat, nisl at consequat
rhoncus, sem nunc gravida justo, quis eleifend arcu velit quis
lacus. Morbi magna magna, tincidunt a, mattis non, imperdiet
vitae, tellus. Sed odio est, auctor ac, sollicitudin in,
consequat vitae, orci. Fusce id felis. Vivamus sollicitudin metus
eget eros.\endgraf
}

Lorem ipsum dolor sit amet, consectetuer adipiscing elit. Nam
cursus. Morbi ut mi. Nullam enim leo, egestas id, condimentum at,
laoreet mattis, massa. Sed eleifend nonummy diam. Praesent mauris
ante, elementum et, bibendum at, posuere sit amet, nibh. Duis
tincidunt lectus quis dui viverra vestibulum. Suspendisse
vulputate aliquam dui. Nulla elementum dui ut augue. Aliquam
vehicula mi at mauris. Maecenas placerat, nisl at consequat
rhoncus, sem nunc gravida justo, quis eleifend arcu velit quis
lacus. Morbi magna magna, tincidunt a, mattis non, imperdiet
vitae, tellus. Sed odio est, auctor ac, sollicitudin in,
consequat vitae, orci. Fusce id felis. Vivamus sollicitudin metus
eget eros.
\end{VerbatimOut}

\MyIOS


\begin{VerbatimOut}{z.out}
\section{Multiple Columns}

Depending on what version of \LaTeX\ you're running
the \verb+multicols+ package may or may not do what
you want.

% The multicols package must be loaded for this to work.
% To load the multicols package put
%     \usepackage{multicols}
% between the "\documentclass" and "\begin{document}" commands.

% Put this amount of space between the columns.
% Let's use the default column separation to see what happens.
% \setlength{\columnsep}{0.5truein}

% Separate the columns with a vertical rule this wide.
% Make the column three times the default width.
\setlength{\columnseprule}{1.2pt}

This is one column\MyRepeat{This is one column.  }{10}

\begin{multicols}{2}
  \MyRepeat{This is two columns.  }{12}
\end{multicols}

\begin{multicols}{3}
  \MyRepeat{This is three columns.  }{9}
\end{multicols}

\begin{multicols}{4}
  \MyRepeat{This is four columns.  }{10}
\end{multicols}

\begin{multicols}{5}
  \MyRepeat{This is five columns.  }{10}
\end{multicols}
\end{VerbatimOut}

\MyIOS


\begin{VerbatimOut}{z.out}


\section{Words}

\newenvironment{entry}
{%
  \bigskip
  % Start a \vbox here.
  % Everything in a \vbox is guaranteed to be on the same page.
  \vbox\bgroup
    \noindent
}
{%
  % End the \vbox here.
  \egroup
}

\begin{entry}
  {\bfseries irregardless}\qquad
  is a nonstandard word that means regardless.
  Use \emph{regardless} instead
  \cite{merriam-webster-irregardless}.
\end{entry}

\begin{entry}
  {\bfseries out of date / out-of-date}\qquad
  means ``outmoded, obsolete''.
  \cite{merriam-webster-out-of-date}.

  When it comes after the noun,
  the compound adjective usually doesn't get a hyphen.
  So we say an easy-to-remember number,
  but the number is easy to remember.
  Same goes for up to date---if it's before a noun it needs a hyphen.
  A document is up to date but it's an up-to-date document
  \cite{thewriter-to-hyphenate-or-not-to-hyphenate}.
  Also see
  \cite{oed-out-of-date}.

  In the context of writing about out-of-date software you may want to
  use ``deprecated'' \cite{merriam-webster-deprecated} instead.
\end{entry}

\begin{entry}
  {\bfseries start-up / start-up company}\qquad
  means a fledgling business enterprise
  \cite{wikipedia-startup-company}.
  I would use the more modern \emph{startup}
  and only use \emph{company} if not clear from the context.
\end{entry}

\begin{entry}
  {\bfseries peace out}\qquad
  means
  ``goodbye''
  \cite{online-slang-dictionary-peace-out}
\end{entry}
\end{VerbatimOut}

\MyIOS
