\ProvidesFile{ap-mathematics.tex}[2021-02-10 mathematics appendix]

\begin{VerbatimOut}{z.out}
\chapter{MATHEMATICS}

\PurdueThesis\ loads the \amsmath\ package \cite{amslatex3project2019}
to do mathematics.
\end{VerbatimOut}

\MyIO


\begin{VerbatimOut}{z.out}
There are two types of mathematics in \LaTeX.
Text math is math that that is interspersed with text.
For example, this is text math: \(a = b + c\).
This is display math:
\begin{equation}
  a = b + c
\end{equation}
\end{VerbatimOut}

\MyIO


\begin{VerbatimOut}{z.out}
\newpage

\section{Standard Functions}

Standard functions should be in a roman font.
Like this: \(\cos\theta\).

Here is a list of standard function commands:\\

\hbox to\hsize{%
  \indent
  \vbox{
    \begin{tabular}{@{}llll@{}}
      \verb+\arccos+& \verb+\csc+& \verb+\ker+&    \verb+\min+\\
      \verb+\arcsin+& \verb+\deg+& \verb+\lg+&     \verb+\Pr+\\
      \verb+\arctan+& \verb+\det+& \verb+\lim+&    \verb+\sec+\\
      \verb+\arg+&    \verb+\dim+& \verb+\liminf+& \verb+\sin+\\
      \verb+\cos+&    \verb+\exp+& \verb+\limsup+& \verb+\sinh+\\
      \verb+\cosh+&   \verb+\gcd+& \verb+\ln+&     \verb+\sup+\\
      \verb+\cot+&    \verb+\hom+& \verb+\log+&    \verb+\tan+\\
      \verb+\coth+&   \verb+\inf+& \verb+\max+&    \verb+\tanh+\\
    \end{tabular}
  }
}
\end{VerbatimOut}

\MyIO


\begin{VerbatimOut}{z.out}
\newpage

\section{English Words in Math}

English words in math should be in a roman font like this:\\
Let the maximum value of \(a\) be \(a_\text{max}\).\\
\(a_\text{max} \ge a_\text{min}\) should always be true.\\
The temperature in the attic is \(t_\text{attic}\).
\end{VerbatimOut}

\MyIO


\begin{VerbatimOut}{z.out}
\section{Text Math}

Use \verb+\(+ to start text math and \verb+\)+ to end text math.
Some people use \verb+$+ to start and end text math---I don't
recommend that because \LaTeX\ can give better error messages
if you use \verb+\(+ and \verb+\)+.
\end{VerbatimOut}

\MyIO


\begin{VerbatimOut}{z.out}
\section{Displayed Equations}

Do not use \verb+$$+ to start or end displayed math like \TeX\ uses
\cite{gratzer2016}.

The \amsmath\ package provides a number
of additional displayed equation structures
beyond the ones provided in basic \LaTeX.
The augmented set includes \cite{amslatex3project2019b}:

\hbox to\hsize{%
  \indent
  \vbox{
    \begin{tabular}{@{}ll@{}}
      \toprule
      \bfseries Environment& \bfseries Used for\\
      \midrule
      \tt equation& used for single equations\\
      \tt multline& split single equations over multiple lines\\
      \tt gather& collect but do not align multiple equations\\
      \tt align& align multiple equations\\
      \tt alignat& aligns multiple equations at multiple places\\
      \tt flalign& aligns multiple equations at multiple places on full length lines\\
      \tt split& split a single equation over multiple lines\\
      \bottomrule
    \end{tabular}
  }%
}

All but \verb+split+ can be followed by \verb+*+ to not number equations.
\end{VerbatimOut}

\MyIOT


\begin{VerbatimOut}{z.out}

\subsection{\texttt{equation} environment}

The \verb+equation+ environment is used for single equations.

\begin{equation}
  E = mc^2
\end{equation}
\end{VerbatimOut}

\MyIOT


\begin{VerbatimOut}{z.out}

The \verb+equation*+ environment does single, unnumbered equations.

\begin{equation*}
  a = b_0c + \frac12 de^2 + {\textstyle \frac12} fg^2
    + h_1 + h_2 + \cdots + h_n
    \qquad \text{for \(c \ne d\) and \( g < \infty\)}
\end{equation*}
\end{VerbatimOut}

\MyIOT


\begin{VerbatimOut}{z.out}

International standard ISO 80000-2:2019 \cite{iso8000022019}
states that $\mit e$,~$\mit i$,~$\mit j$,
and $\itpi$ should appear as
$e$,~$i$,~$j$
and~$\pi$ because they are constants.
This is done automatically by the mismath package
that is loaded by thesis.tex.
See thesis.tex for more information,
including what to do if you're not using those as constants.

Euler's identity is
\begin{equation*}
  e^{i\pi} + 1 = 0.
\end{equation*}
\end{VerbatimOut}

\MyIOT


\begin{VerbatimOut}{z.out}

Here's a simple formula relating $e$,~$i$,~$\pi$, and~$\phi$,
the golden ratio
\begin{equation}
  e^{i\pi} + 2\phi = \sqrt 5.
\end{equation}
I didn't notice anything on the web about putting the symbol for
the golden ratio in a special font even though it is a constant.
\end{VerbatimOut}

\MyIOT


\begin{VerbatimOut}{z.out}
  
International standard ISO 80000-2:2019 \cite{iso8000022019}
states that the ``$d\/$'' in math differentials
should be typeset as ``$\di$''.
So,
\begin{equation*}
  \text{use } \int x\di x\qquad\qquad \text{instead of } \int x\,dx
\end{equation*}
\end{VerbatimOut}

\MyIOT


\begin{VerbatimOut}{z.out}

The formula for Bekenstein-Hawking entropy:

\begin{equation*}
  S_\text{BH} = \frac A{4L_P^2} = \frac{c^3A}{4G\hbar}
\end{equation*}
\end{VerbatimOut}

\MyIOT


\begin{VerbatimOut}{z.out}

Type in the math and let \LaTeX\ worry about the spacing.
You only need to do fine tuning by hand if it looks bad.  

Another \verb+equation*+ environment,
note the spacing before the large close parenthesis:

\begin{equation*}
  \frac ab = ab^{-1}
    % Parens are the wrong size.
    = (\sqrt{\frac ab})^2
    % Parens are the right size but closing paren is too close to radical.
    = \left( \sqrt\frac ab \right)^2
    % Parens are right size but a negative thin space puts closing paren too close to radical.
    = \left( \sqrt\frac ab \!\right)^2
    % Parens are right size but a thin space puts closing paren too close to radical.
    = \left( \sqrt\frac ab \,\right)^2
    % Parens are right size but a medium space puts closing paren too close to radical.
    = \left( \sqrt\frac ab \:\right)^2
    % Parens are right size and I think a thick space looks the best.
    = \left( \sqrt\frac ab \;\right)^2
\end{equation*}
\end{VerbatimOut}

\MyIOT


\begin{VerbatimOut}{z.out}

\begin{equation*}
  (\cos x)^2 + (\sin x)^2 = \cos^2 x + \sin^2 x = 1
\end{equation*}
\end{VerbatimOut}

\MyIOT


\begin{VerbatimOut}{z.out}

\begin{equation}
  x \mod 2 =
  \begin{cases}
    0& \text{if $x$ is even}\\
    1& \text{if $x$ is odd}\\
  \end{cases}
\end{equation}
\end{VerbatimOut}

\MyIOT


\begin{VerbatimOut}{z.out}

The first six derivatives of distance are velocity, acceleration, jerk, snap, crackle, and pop \cite{reid2013}.

\begin{equation}
  % Every array element should be in \displaystyle (a big font).
  \AtBeginEnvironment{array}{\everymath{\displaystyle}}
  % Set space between columns to zero, use {} = ... to add a little space before the = "by hand".
  \arraycolsep = 0pt
  \text{distance derivitives} = \left\{\ %
    \begin{array}{llllllll}
      % I'm formatting the first 4 lines different from the last 3 so this will fit on one page.
      x&      {}=\text{distance}&     {}=vt\\[2pt]
      v&      {}=\text{velocity}&     {}=\frac{\di x}{\di t}\\[9pt]
      a&      {}=\text{acceleration}& {}=\frac{\di v}{\di t}& {}=\frac{\di^2x}{\di t^2}\\[9pt]
      \mit j& {}=\text{jerk}&         {}=\frac{\di a}{\di t}& {}=\frac{\di^2v}{\di t^2}&
        {}=\frac{\di^3x}{\di t^3}\\[9pt]
      s
        & {}=\text{snap}
        & {}=\frac{\di \mit j}{\di t}
        & {}=\frac{\di^2a}{\di t^2}
        & {}=\frac{\di^3v}{\di t^3}
        & {}=\frac{\di^4x}{\di t^4}\\[9pt]
      c
        & {}=\text{crackle}
        & {}=\frac{\di s}{\di t}
        & {}=\frac{\di^2\mit j}{\di t^2}
        & {}=\frac{\di^3a}{\di t^3}
        & {}=\frac{\di^4v}{\di t^4}
        & {}=\frac{\di^5x}{\di t^5}\\[9pt]
      % 
      p
        & {}=\text{pop}
        & {}=\frac{\di c}{\di t}
        & {}=\frac{\di^2s}{\di t^2}
        & {}=\frac{\di^3\mit j}{\di t^3}
        & {}=\frac{\di^4a}{\di t^4}
        & {}=\frac{\di^5v}{\di t^5}
        & {}=\frac{\di^6x}{\di t^6}
    \end{array}
  \right.
\end{equation}
\end{VerbatimOut}

\MyIOT


\begin{VerbatimOut}{z.out}

\subsection{\texttt{multline} environment}

The \verb+multline+ environment is used
to split single equations over multiple lines.

\begin{multline}
  S = a + b + c + d + e + f + g + h + i + j\\
  + k + l + m + n + o + p\\
  + q + r + s + t + u + v + w + x + y + z
\end{multline}
\end{VerbatimOut}

\MyIOT


\begin{VerbatimOut}{z.out}

\begin{multline}
  S = a + b + c + d + e\\
  + f + g + h + i + j\\
  + k + l + m + n + o\\
  + p + q + r + s + t\\
  + u + v + w + x + y\\
  + z
\end{multline}
\end{VerbatimOut}

\MyIOT


\begin{VerbatimOut}{z.out}

% Calculate width of space before equation plus equation number.
% (All digits are the same width.)  
\newdimen{\tdimen}
\settowidth{\tdimen}{\kern\multlinetaggap (L.5)}
\begin{multline}
  S = a + b + c + d + e\\
  \makebox[\textwidth]{\hfill $+ f + g + h + i + j$\hfill\hfill\hfill\hfill\kern\tdimen}\\
  \makebox[\textwidth]{\hfill\hfill${} + k + l + m + n + o$\hfill\hfill\hfill\kern\tdimen}\\
  \makebox[\textwidth]{\hfill\hfill\hfill${} + p + q + r + s + t$\hfill\hfill\kern\tdimen}\\
  \makebox[\textwidth]{\hfill\hfill\hfill\hfill${} + u + v + w + x + y$\hfill\kern\tdimen}\\
  + z
\end{multline}
\end{VerbatimOut}

\MyIOT


\begin{VerbatimOut}{z.out}

\subsection{\texttt{gather} environment}

The \verb+gather+ environment collects but does not align multiple equations.

\begin{gather}
  a = b + c + d + e + f + g + h + i + j + k + l\\
  m = n + o + p + q + r + s + t + u + v + w + x + y + z
\end{gather}
\end{VerbatimOut}

\MyIO


\begin{VerbatimOut}{z.out}

\begin{gather}
  a = b + c + d + e + f + g + h + i + j + k + l\notag\\
  m = n + o + p + q + r + s + t + u + v + w + x + y + z
\end{gather}
\end{VerbatimOut}

\MyIO


\begin{VerbatimOut}{z.out}

\begin{gather*}
  \alpha = \beta + \gamma + \delta + \eta\\
  \theta = \iota + \kappa + \lambda + \mu + \nu + \rho + \tau
\end{gather*}
\end{VerbatimOut}

\MyIO


\begin{VerbatimOut}{z.out}

\begin{gather}
  x_\text{min} + x_\text{max} \le \sum_{i=1}^n x_i\\
  x_\text{min} + x_\text{max}
    = \sum_{i=1}^n x_i - \sum_{i=2}^{n-1} x_i\quad\text{if $x$ is sorted}\\
  x_\text{min} \le \left(\sum_{i=1}^n x_i\right) / n
\end{gather}
\end{VerbatimOut}

\MyIOT


\begin{VerbatimOut}{z.out}

\subsection{\texttt{align} environment}

The \verb+align+ environment aligns multiple equations.

\begin{align}
  a &= b + c + d\\
  e &= f + g + h + i + j
\end{align}
\end{VerbatimOut}

\MyIO


\begin{VerbatimOut}{z.out}

\begin{align}
  x = \frac{-b \pm \sqrt{b^2-4ac}}{2a}\notag\\
  % Put a thin space before the b^2 to improve the appearance.
  x = \frac{-b \pm \sqrt{\,b^2-4ac}}{2a}
\end{align}
\end{VerbatimOut}
\index{align}
\index{\verb+\begin{gather}+}
\index{thin space}
\index{\verb+\,+}

\MyIOT


\begin{VerbatimOut}{z.out}
  
Quadratic formula proof \cite{khan}:

% The align environment requires the amsmath package.
% Use \addtolength{\jot}{6pt} to increase the space between rows in an amsmath multi-line math formula.
% That's not done here so everything will fit on one page.
\begin{align}
  ax^2 + bx + c &= 0\\
  ax^2 + bx &= -c\notag\\
  % The "\," adds a thinspace of horizontal space.
  x^2 + \frac ba\,x &= -\frac ca\notag\\
  x^2 + \frac ba\,x + \frac{b^2}{4a^2} &= \frac{b^2}{4a^2} - \frac ca\notag\\
  \left(x + \frac b{2a}\right)^2 &= \frac{b^2}{4a^2} - \frac ca\notag\\
  \left(x + \frac b{2a}\right)^2 &= \frac{b^2}{4a^2} - \frac{4ac}{4a^2}\notag\\
  \left(x + \frac b{2a}\right)^2 &= \frac{b^2-4ac}{4a^2}\notag\\
  \sqrt{\left(x + \frac b{2a}\right)^2}
    &= \sqrt{\left(\frac{b^2-4ac}{4a^2}\right)}\notag\\
  x + \frac b{2a} &= \pm \frac{\sqrt{\,b^2-4ac}}{\sqrt{4a^2}}\notag\\
  x + \frac b{2a} &= \pm \frac{\sqrt{\,b^2-4ac}}{2a}\notag\\
  x &= - \frac b{2a} \pm \frac{\sqrt{\,b^2-4ac}}{2a}\notag\\
  x &= \frac{-b \pm \sqrt{\,b^2-4ac}}{2a}
\end{align}
\end{VerbatimOut}

\MyIOT


\begin{VerbatimOut}{z.out}

\subsection{\texttt{alignat} environment}

The \verb+alignat+ environment aligns multiple equations at multiple places.
\begin{alignat}{2}
  a &= b& \qquad\qquad& \text{set $a$}\\
  c &= d& &             \text{you guessed it, set $c$}\notag\\
  g &= h& &             \text{and finally, set $g$}
\end{alignat}

I like to align input columns on the input if possible
and will sometimes use windows over~250 characters wide to do so.
If that won't work I sometimes do,
for example,
\begin{alignat}{2}
  a
    &= b
    & \qquad\qquad
    & \text{set $a$}\\
  c
    &= d
    &
    &\text{you guessed it, set $c$}\notag\\
  g
    &= h
    &
    &\text{and finally, set $g$}
\end{alignat}

Do whatever works best for you.

\end{VerbatimOut}

\MyIOT


\begin{VerbatimOut}{z.out}
  
Quadratic formula proof \cite{khan}:

% Make changes to \jot be local to the group that starts on the next line.
{
  % Increase distance between lines by 6pt.
  \addtolength{\jot}{6pt}
  \begin{alignat}{2}
    ax^2 + bx + c
      &= 0
      &
      &\text{subtract $c$}\\
    ax^2 + bx
      &= -c
      &
      &\text{divide by $a$}\notag\\
    % The "\," adds a thinspace of horizontal space.
    x^2 + \frac ba\,x
      &= -\frac ca
      &
      &\text{add $\displaystyle\frac{b^2}{4a^2}$}\notag\\
    x^2+\frac ba\,x+\frac{b^2}{4a^2}
      &= \frac{b^2}{4a^2}-\frac ca
      &
      &\text{factor left hand side}\notag\\
    \left(x+\frac b{2a}\right)^2
      &= \frac{b^2}{4a^2}-\frac ca
      &
      &\text{multiply right-most term by $\displaystyle\frac{4a}{4a}$}\notag\\
    \left(x + \frac b{2a}\right)^2
      &= \frac{b^2}{4a^2}-\frac{4ac}{4a^2}
      &
      &\text{use common denominator}\notag\\
    \left(x + \frac b{2a}\right)^2
      &= \frac{b^2-4ac}{4a^2}
      &
      &\text{take square root of each side}\notag\\
    \sqrt{\left(x + \frac b{2a}\right)^2}
      &= \sqrt{\left(\frac{b^2-4ac}{4a^2}\right)}
      &
      &\text{compute square root of each side}\notag\\
    x + \frac b{2a}
      &= \pm \frac{\sqrt{\,b^2-4ac}}{\sqrt{4a^2}}
      &
      &\text{simplify right hand denominator}\notag\\
    x + \frac b{2a}
      &= \pm \frac{\sqrt{\,b^2-4ac}}{2a}
      &
      &\text{subtract $\displaystyle\frac b{2a}$ from each side}\notag\\
    x
      &= -\frac b{2a} \pm \frac{\sqrt{\,b^2-4ac}}{2a}
      &\qquad
      &\text{use common denominator}\notag\\
    x
      &= \frac{-b \pm \sqrt{\,b^2-4ac}}{2a}
  \end{alignat}
}
\end{VerbatimOut}

\MyIOS


\begin{VerbatimOut}{z.out}

\subsection{\texttt{flalign} environment}

The \verb+flalign+ environment aligns multiple equations at multiple places
on full length lines.

\begin{flalign}
  a &= b&   &   & u &= v\\
  c &= d& m &= n& w &= x\notag\\
  g &= h&   &   & y &= z
\end{flalign}
\end{VerbatimOut}

\MyIOT


\begin{VerbatimOut}{z.out}

\subsection{\texttt{split} environment}

The \verb+split+ environment ???.
\end{VerbatimOut}

\MyIOT


\begin{VerbatimOut}{z.out}
\section{Theorem-like environments}

These theorem-like environments are defined
in the amsthm package or in \verb+PurdueThesis.cls+.
\end{VerbatimOut}

\MyIOT


\begin{VerbatimOut}{z.out}

\begin{definition}
  This is an example definition.
\end{definition}

\begin{observation}
  This is an example observation.
\end{observation}

\begin{proof}
  This is an example proof.
  If \(a = b\) and \(b = c\) then \(a = c\).
\end{proof}

\begin{proposition}
  This is an example proposition.
\end{proposition}

\begin{theorem}
  This is an example theorem.
\end{theorem}
\end{VerbatimOut}

\MyIOT


\begin{VerbatimOut}{z.out}
brackets and stuff

\href{https://en.wikipedia.org/wiki/Squared_triangular_number}{Nicomachus's theorem}:
\begin{equation}
  \sum_{j=1}^n j^3 = \left(\sum_{j=1}^n j\right)^2
\end{equation}

Also see the
\href{proof without words}{https://mathworld.wolfram.com/ProofwithoutWords.html}
on that same page.
\end{VerbatimOut}

\MyIOT







% https://www.google.com/url?sa=i&rct=j&q=&esrc=s&source=images&cd=&cad=rja&uact=8&ved=0ahUKEwiSzf370-jmAhXIGs0KHRHyAZQQMwiCASgNMA0&url=https%3A%2F%2Fwww.chegg.com%2Fhomework-help%2Fquestions-and-answers%2Fequations-displacement-atoms-along-linear-chain-d2u-u-m-mass-atoms-c-force-constant-neares-q26052186&psig=AOvVaw1eYq-Q0El0Tbjgbp_Lu2Vv&ust=1578183004388936&ictx=3&uact=3

% M\frac{d^2\mu_n}{dt^2} = C(u_{n_1} - 2u_n + u_{n-1}).

% a^{b} same as a^b     c^{de} f^{g^h} ij^{kl}^{mn}

% superscripts
% subscripts

% operators

%\sum
%\prod

%\alpha
%\beta
%\gamma
%\delta
%\epsilor
%\varepsilon
%\zero
%\eta
%\theta
%\vartheta
%\iota
%\kappa
%\lambda
%\mu
%\nu\xi
%\varrho
%\sigma
%\varsigma
%\tau
%\upsilan
%\pha
%\varphi
%\chi
%\psi
%\omega

%\Gamma
%\Delta
%\Theta
%\Lambda
%\Xi
%\Pi
%\Sigmo
%\Upsilan
%\Phi
%\Psi
%\Omega

%mixed greek and text

%x_min, x_max

%\ldots

%\to

%1 + \frac14 + \frac19 + \cdots = \frac\pi6

%fractions

% frac{\ln x}{\ln\alpha} = log_\alpha x

% \sum_{i=1}^n = 1 + 2 + \cdots + n = \frac{n(n+1)}{2}


% xxxx     \end{verbatim}
% xxxx % Requires \usepackage{amsmath}; use align* for no equation number.
% xxxx \begin{align}
% xxxx   a = {}& b + c\\
% xxxx   x = {}& y + z
% xxxx \end{align}
% xxxx     \vskip\baselineskip
% xxxx     \hrule
% xxxx     \vskip0.5\baselineskip
% xxxx     \filbreak
% xxxx 
% xxxx     \begin{verbatim}
% xxxx \[
% xxxx   Z =
% xxxx     \left(
% xxxx       \begin{array}{cc}
% xxxx         a& b\\
% xxxx         c& d
% xxxx       \end{array}
% xxxx     \right)
% xxxx \]
% xxxx     \end{verbatim}
% xxxx \[
% xxxx   Z =
% xxxx     \left(
% xxxx       \begin{array}{cc}
% xxxx         a& b\\
% xxxx         c& d
% xxxx       \end{array}
% xxxx     \right)
% xxxx \]
% xxxx     \vskip\baselineskip
% xxxx     \hrule
% xxxx     \vskip0.5\baselineskip
% xxxx     \filbreak
% xxxx 
% xxxx     \begin{verbatim}
% xxxx \begin{equation}
% xxxx   \begin{split}
% xxxx     a = {}& b + c\\
% xxxx       & {} + d + e
% xxxx   \end{split}
% xxxx \end{equation}
% xxxx     \end{verbatim}
% xxxx \begin{equation}
% xxxx   \begin{split}
% xxxx     a = {}& b + c\\
% xxxx       & {} + d + e
% xxxx     \end{split}
% xxxx \end{equation}
% xxxx     \vskip\baselineskip
% xxxx     \hrule
% xxxx     \vskip0.5\baselineskip
% xxxx     \filbreak
% xxxx 
% xxxx     \begin{verbatim}
% xxxx \[
% xxxx   (\cos x)^2 + (\sin x)^2 = 1
% xxxx \]
% xxxx     \end{verbatim}
% xxxx \[
% xxxx   (\cos x)^2 + (\sin x)^2 = 1
% xxxx \]
% xxxx     \vskip\baselineskip
% xxxx     \hrule
% xxxx     \vskip0.5\baselineskip
% xxxx     \filbreak
% xxxx 
% xxxx     \begin{verbatim}
% xxxx If $X = \cos x$ and $Y = \sin x$ then $X^2 + Y^2 = 1$.
% xxxx     \end{verbatim}
% xxxx If $X = \cos x$ and $Y = \sin x$ then $X^2 + Y^2 = 1$.
% xxxx     \vskip\baselineskip
% xxxx     \hrule
% xxxx     \vskip0.5\baselineskip
% xxxx     \filbreak
% xxxx 
% xxxx 


