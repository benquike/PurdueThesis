\ProvidesFile{ap-chemistry.tex}[2021-01-04 chemistry appendix]

\begin{VerbatimOut}{z.out}
\chapter{CHEMISTRY}
\end{VerbatimOut}

\MyIOT


\begin{VerbatimOut}{z.out}

\section{Chemical Diagrams}

The chemplants package \cite{feffin2019}
extends the
\href{http://ctan.math.washington.edu/tex-archive/graphics/pgf/base/doc/pgfmanual.pdf}{\TikZ} package
to draw chemical process units.
\end{VerbatimOut}

\MyIO


\begin{VerbatimOut}{z.out}

\section{Chemical Equations}

The mhchem Bundle \cite{hensel2018}
contains mhchem v4.08 (chemical equations),
hpstatement v1.02 (official hazard and precautionary statements),
and rsphrase v3.11 (official rist and safety phrases).
\end{VerbatimOut}

\MyIO




\begin{VerbatimOut}{z.out}

Defined in thesis.tex: \nitrate.
\end{VerbatimOut}

\MyIOT


\begin{VerbatimOut}{z.out}

% See page 1 of
%     https://www.thoughtco.com/what-is-a-chemical-equation-604026
\ce{CH4 + 2O2 -> CO2 + 2H2O}
\end{VerbatimOut}

\MyIOT


\begin{VerbatimOut}{z.out}

% See page 1 of
%     https://www.thoughtco.com/what-is-a-chemical-equation-604026
\ce{2H2(g) + O2(g) -> 2H2O(l)}
\end{VerbatimOut}

\MyIOT


\begin{VerbatimOut}{z.out}

% See page 1 of
% https://www.thoughtco.com/definition-of-ionic-equation-605262
\ce{Ag+(aq) + NO3-(aq) + Na+(aq) + Cl-(aq) -> AgCl(s) + Na+(aq) + NO3-(aq)}
is an ionic equation of the chemical reaction:
\ce{AgNO3(aq) + NaCl(aq) -> AgCl(s) + NaNO3(aq)}
\end{VerbatimOut}

\MyIOT


\begin{VerbatimOut}{z.out}

% See page 1 of
%     https://www.thoughtco.com/definition-of-balanced-equation-and-examples-604380
\ce{Fe2O3 + C -> Fe + CO2}
\end{VerbatimOut}

\MyIOT


\begin{VerbatimOut}{z.out}

% From page 1 of
%     https://www.thoughtco.com/definition-of-molecular-equation-605366

For example, in the reaction between sodium chloride
(\ce{NaCl})
and silver nitrate
(\ce{AgNO3}),
the molecular reaction is:

\ce{NaCl(aq) + AgNO3 -> NaNO3(aq) + AgCl(s)}
\end{VerbatimOut}

\MyIOT


\begin{VerbatimOut}{z.out}

The complete ionic equation is:

\ce{Na+(aq) + Cl-(aq) + Ag+(aq) + NO3-(aq) -> AgCl(s) + Na+(aq) + NO3-(aq)}
\end{VerbatimOut}

\MyIOT


\begin{VerbatimOut}{z.out}

Ruben Meerman \cite[starting at 5:25]{meerman} claims this equation
\begin{center}
  \ce{C55H104O6 + 78O2 -> 55CO2 + 52H2O + energy}\endgraf
\end{center}
describes weight loss.
\end{VerbatimOut}

\MyIOT

\begin{VerbatimOut}{z.out}

And with better annotation:

\begin{center}
  \newcommand{\vph}{{\vphantom{\large Ag}}}
  \ce{
    $\underset{\text{\vph \footnotesize human fat}}{\ce{C55H104O6}}$
    +
    $\underset{\text{\vph \footnotesize oxygen}}{\ce{78O2}}$
    ->
    $\underset{\text{\vph \footnotesize carbon dioxide}}{\ce{55CO2}}$
    +
    $\underset{\text{\vph \footnotesize water}}{\ce{52H2O}}$
    +
    $\underset{\text{\vph \footnotesize body heat, moving, thinking, growing}}{\text{energy}}$
  }
\end{center}
\end{VerbatimOut}

\MyIOT

\begin{VerbatimOut}{z.out}

And with still better annotation:

\begin{center}
  \newcommand{\Fs}{\scriptsize}
  \begin{tabular}{@{}c@{}c@{}c@{}c@{}c@{}c@{}c@{}c@{}c@{}}
    &                                                       %  1. C55H10406
      &                                                     %  2. +
      &                                                     %  3. 78O2
      &                                                     %  4. ->
      &                                                     %  5. 55CO2
      &                                                     %  6. +
      &                                                     %  7. 52H2O
      &                                                     %  8. +
      \Fs calories\\                                        %  9. energy
    %
    \Fs kg&                                                 %  1.
      &                                                     %  2.
      \Fs kg&                                               %  3.
      &                                                     %  4.
      \Fs kg&                                               %  5.
      &                                                     %  6.
      \Fs kg&                                               %  7.
      &                                                     %  8.
      \Fs kJ\\                                              %  9.
    %
    \noalign{\vspace{3pt}}
    %
    \ce{C55H104O6}                                          %  1.
      & \ce{+}                                              %  2.
      & \ce{78O2}                                           %  3.
      & \ce{->}                                             %  4.
      & \ce{55CO2}                                          %  5.
      & \ce{+}                                              %  6.
      & \ce{52H2O}                                          %  7.
      & \ce{+}                                              %  8.
      & energy\\                                            %  9.
    %
    \Fs human fat&                                          %  1.
      &                                                     %  2.
      \Fs oxygen&                                           %  3.
      &                                                     %  4.
      \Fs carbon dioxide&                                   %  5.
      &                                                     %  6.
      \Fs water&                                            %  7.
      &                                                     %  8.
      \Fs body heat, moving, thinking, growing\\            %  9.
  \end{tabular}
\end{center}
\end{VerbatimOut}

\MyIOT


\begin{VerbatimOut}{z.out}


\section{Chemical Figures}

Below is an example of how to use the chemfig package \cite{tellechea2019}.

% Chicago Manual of Style Online, 17 edition, section 9.61 states
% that 72--73, not 72--3, should be used.
Here is the chemical figure
for Penicillin \cite[pages~72--73]{tellechea2019}:\\

\chemfig{[:-90]HN(-[::-45](-[::-45]R)=[::+45]O)>[::+45]*4(-(=O)-N*5(-(<:(=[::-60]O)
-[::+60]OH)-(<[::+0])(<:[::-108])-S>)--)}
\end{VerbatimOut}

\MyIOT
