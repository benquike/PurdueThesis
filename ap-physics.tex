\ProvidesFile{ap-physics.tex}[2021-01-04 physics appendix]

\begin{VerbatimOut}{z.out}
\chapter{PHYSICS}

Feynman diagrams show what happens
when elementary particles collide
\cite{feynman-diagram}.
The Feynman diagrams below are from the
\citetitle{ellis2016} documentation \cite{ellis2016}.
\textbf{%
  You must use \texttt{lualatex} instead
  of \texttt{pdflatex}
  to process documents that use the \texttt{tikz-feynman} package.%
}
See the documentation for more information.

The input
in the documentation
is different than here because a different random number generator
is used \cite{menke2019}.
I expect this to be corrected.
In the meantime try replacing \texttt{vertical}
with \texttt{vertical'}
and/or switch some \texttt{fermion}
to \texttt{anti} \texttt{fermion} lines \cite{ellis2017}.
\end{VerbatimOut}

\MyIOT

\begin{VerbatimOut}{z.out}
\feynmandiagram [large, vertical'=e to f] {
  a -- [fermion] b -- [photon, momentum=\(k\)] c -- [fermion] d,
  b -- [fermion, momentum'=\(p_{1}\)] e -- [fermion, momentum'=\(p_{2}\)] c,
  e -- [gluon] f,
  h -- [anti fermion] f -- [anti fermion] i,
};
\end{VerbatimOut}

\MyIOT

\begin{VerbatimOut}{z.out}
\feynmandiagram [horizontal=a to b] {
  i1 -- [anti fermion] a -- [anti fermion] i2,
  a -- [photon] b,
  f1 -- [fermion] b -- [fermion] f2,
};
\end{VerbatimOut}

\MyIOT
